\documentclass[12pt]{article}
\usepackage[margin=1in]{geometry}
\usepackage{setspace}
\usepackage{graphicx}
\usepackage{float}
\usepackage{amsmath}
\usepackage{booktabs}
\usepackage{hyperref}

\title{The Association Between the Kaitz Index and Employment in Puerto Rico}
\author{
  Alejandro M. Ouslan \\
  \small{University of Puerto Rico, Mayaguez} \\
  \small{\texttt{alejandro.ouslan@upr.edu}}
  \and
  Julio C. Hernández \\
  \small{University of Puerto Rico, Mayaguez} \\
  \small{\texttt{julio.hernandez3@upr.edu}}
}
\date{May 14, 2025}

\begin{document}

\maketitle


\begin{abstract}
	\textit{
		This study examines the relationship between the Kaitz Index and employment levels in Puerto Rico using panel data at the
		postal code level. The data covers the period from Q1 2012 to Q4 2023. A Spatial Durbin Model with fixed effects is applied
		to investigate both the direct effect of the Kaitz Index on employment and potential spatial effects from adjacent regions.
		The findings largely support existing literature suggesting a negative relationship between the minimum wage and employment.
		Additionally, despite some convergence issues with the MCMC of the Kaitz variable, the confidence interval indicates a
		negative spillover effect on neighboring regions’ employment levels.
	}
\end{abstract}

\end{document}
